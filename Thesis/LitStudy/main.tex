\documentclass[a4paper,12pt]{article}
\usepackage[dutch]{babel}
\usepackage{amsmath}
\usepackage{amssymb}
\usepackage{graphicx}
\usepackage{siunitx}
\usepackage{parskip}
\usepackage{hyperref}
\usepackage{lipsum}
\usepackage[fixlanguage]{babelbib}
\usepackage{todonotes}
\everymath={\displaystyle}
\renewcommand{\figureautorefname}{Figuur}
\renewcommand{\tableautorefname}{Tabel}
\renewcommand{\partautorefname}{Deel}
\renewcommand{\appendixautorefname}{Appendix}
\renewcommand{\partautorefname}{Vergelijking}
\renewcommand{\Itemautorefname}{punt}
\renewcommand{\chapterautorefname}{Hoofdstuk}
\renewcommand{\sectionautorefname}{Sectie}
\renewcommand{\subsectionautorefname}{Sectie}
\renewcommand{\subsubsectionautorefname}{Sectie}
\renewcommand{\paragraphautorefname}{paragraaf}
\renewcommand{\Hfootnoteautorefname}{noot}
\renewcommand{\AMSautorefname}{Vergelijking}
\renewcommand{\theoremautorefname}{Stelling}
\renewcommand{\pageautorefname}{pagina}
\interfootnotelinepenalty=10000

\begin{document}
\author{Randy Thiemann} 
\title{Redirected Walking} 
\date{\today} 
\maketitle

%%%%%%%%%%%%%%%%%%%%%%%%%%%%%%%%%%%%%%%%%%%%%%%%%%%%%%%%%%%%%%%%%%%%%%%%%%%%%%%%%
\section{Literatuurstudie}
\subsection{Inleiding}
\emph{Redirected walking} is een techniek waarbij een gebruiker in een virtuele
omgeving weergegeven in een head-mounted display kan rondwandelen door middel
van het vervormen van de route die deze gebruiker in een fysieke omgeving wandelt.

Deze term omvat diverse technieken, ik tracht deze technieken hier in het kort
toe te lichten door middel van eerder uitgevoerde studies.

Vervolgens zal ik enkele inherente zwaktes in redirected walking toelichten.

Ik zal dan besluiten met de invalshoek die ik zal nemen in deze bachelorproef.

%%%%%%%%%%%%%%%%%%%%%%%%%%%%%%%%%%%%%%%%%%%%%%%%%%%%%%%%%%%%%%%%%%%%%%%%%%%%%%%%%
\subsection{Technieken}
\subsubsection{Rotationele vervorming}
In een onderzoek gevoerd door Razzaque, Kohn en Whitton\cite{kohn01} werden 
gebruikers gevraagd om een virtuele brandoefening uit te voeren.

In de opstelling voor dit onderzoek waren er in het fysieke labo twee knoppen
geplaatst op dezelfde afstand als in de virtuele omgeving. In de virtuele
omgeving waren er echter 4 knoppen met telkens een hoek van 90 graden er tussen.

In dit onderzoek heeft men ondervonden dat er 3 manieren zijn om rotationele
vervorming in te voegen:

\begin{enumerate}
    \item Als de gebruiker stil staat is het mogelijk om een kleine hoeveelheid
        constante rotatie in te voegen, de gebruiker zal dan automatisch
        meedraaien.
    \item Indien de gebruiker zelf ronddraait kan deze rotatie overdreven worden.
    \item Ten laatste is het mogelijk om het pad in de virtuele omgeving een
        bepaalde hoeveelheid te buigen die proportioneel is met de lineaire
        snelheid van de gebruiker in de fysieke omgeving.
\end{enumerate}

Deze gegevens werden dan gebruikt om de gebruiker in de richting van de volgende
knop te sturen zodat de gebruiker voor een fysieke knop stond als dit ook het
geval was in de virtuele omgeving.

Om de immersie in de virtuele omgeving te verbeteren werd er ook gebruik gemaakt
van positionele audio met met gesloten koptelefoons die het oor volledig omringen.

Ten laatste werd er om botsingen te voorkomen een soort alarmsysteem ingevoerd
dat gebruiker vroeg om links en rechts te draaien, zodat het systeem zich kon 
hercalibreren.

In een ander onderzoek van Neth et. al.\cite{neth12} werd verder onderzocht wat
de precieze relatie tussen bewegingssnelheid en maximaal acceptabele rotationele
vervorming is.

\todo[inline]{Praten over experiment 1 v. neth12}

Rotationele vervorming vormt de basis van redirected walking, gegeven een
voldoende grote ruimte kan deze techniek toegepast worden om elke virtuele
omgeving te doorlopen.

%%%%%%%%%%%%%%%%%%%%%%%%%%%%%%%%%%%%%%%%%%%%%%%%%%%%%%%%%%%%%%%%%%%%%%%%%%%%%%%%%
\subsection{Inherente zwaktes}
Hoewel redirected een realistische en immersieve omgeving kan cre\"eren zijn er
toch enkele inherente zwaktes waar op zich niet omheen kan gewerkt worden.

Zo is het bijvoorbeeld onmogelijk om arbitraire fysieke collisie overeen te laten
komen met collisie in de virtuele omgeving tenzij het labo expliciet voor die
virtuele omgeving is gebouwd.

Een ander voorbeeld is dat het onmogelijk is om heuvels of ander ruw terrein te
hebben in de virtuele omgeving zonder immersie te breken.

Desondanks deze behoorlijke beperkingen heeft redirected walking toch veel
potenti\"ele toepassingen zoals virtuele rondleidingen.

%%%%%%%%%%%%%%%%%%%%%%%%%%%%%%%%%%%%%%%%%%%%%%%%%%%%%%%%%%%%%%%%%%%%%%%%%%%%%%%%%
\subsection{Verder onderzoek}
\todo[inline]{Mijn BPROEF bespreken.}

\newpage
\bibliographystyle{babplain-fl}
\bibliography{refs}
  
\end{document}