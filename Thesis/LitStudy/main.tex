\documentclass[a4paper,12pt]{article}
\usepackage[dutch]{babel}
\usepackage{amsmath}
\usepackage{amssymb}
\usepackage{graphicx}
\usepackage{siunitx}
\usepackage{parskip}
\usepackage{hyperref}
\usepackage{lipsum}
\usepackage[fixlanguage]{babelbib}
\usepackage{todonotes}
\everymath={\displaystyle}
\renewcommand{\figureautorefname}{Figuur}
\renewcommand{\tableautorefname}{Tabel}
\renewcommand{\partautorefname}{Deel}
\renewcommand{\appendixautorefname}{Appendix}
\renewcommand{\partautorefname}{Vergelijking}
\renewcommand{\Itemautorefname}{punt}
\renewcommand{\chapterautorefname}{Hoofdstuk}
\renewcommand{\sectionautorefname}{Sectie}
\renewcommand{\subsectionautorefname}{Sectie}
\renewcommand{\subsubsectionautorefname}{Sectie}
\renewcommand{\paragraphautorefname}{paragraaf}
\renewcommand{\Hfootnoteautorefname}{noot}
\renewcommand{\AMSautorefname}{Vergelijking}
\renewcommand{\theoremautorefname}{Stelling}
\renewcommand{\pageautorefname}{pagina}
\interfootnotelinepenalty=10000

\begin{document}
\author{Randy Thiemann} 
\title{Redirected Walking} 
\date{\today} 
\maketitle

%%%%%%%%%%%%%%%%%%%%%%%%%%%%%%%%%%%%%%%%%%%%%%%%%%%%%%%%%%%%%%%%%%%%%%%%%%%%%%%%%
\section{Literatuurstudie}
\subsection{Inleiding}
\emph{Redirected walking} is een techniek waarbij een gebruiker in een virtuele
omgeving weergegeven in een head-mounted display kan rondwandelen door middel
van het vervormen van de route die deze gebruiker in een fysieke omgeving wandelt.

Het doel is, door de 1:1 relatie van fysieke beweging en virtuele beweging te
ontkoppelen, rondwandelen in grotere tot zelfs arbitrair grote virtuele ruimtes
mogelijk te maken.

Om dit te effect te bekomen zijn er 2 brede categorie\"en van technieken, ik zal
deze twee categorie\"en beknopt toelichten samen met enkele concrete voorbeelden.

Vervolgens zal ik enkele inherente zwaktes in redirected walking toelichten.

Ik zal dan besluiten met de invalshoek die ik zal nemen in deze bachelorproef.


%%%%%%%%%%%%%%%%%%%%%%%%%%%%%%%%%%%%%%%%%%%%%%%%%%%%%%%%%%%%%%%%%%%%%%%%%%%%%%%%%
\subsection{Redirectietechnieken (RDTs)}
RDTs zijn technieken waar er gebruik wordt gemaakt van de zwaktes van het 
vestibulair systeem om de fysieke bewegingen van een gebruiker te ontkoppelen
van de resulterende virtuele bewegingen.

Bij deze technieken is het belangrijk om de vervormingen onmerkbaar te houden
voor de gebruiker, maar ook om ze consistent te houden om simulatieziekte te
vermijden.\cite{kohn01}

Ik beschrijf hier kort deze technieken samen met onderzoeken die er over gevoerd
zijn.


\subsubsection{Rotationele vervorming}
In een onderzoek gevoerd door Razzaque, Kohn en Whitton\cite{kohn01} werden 
gebruikers gevraagd om een virtuele brandoefening uit te voeren.

In de opstelling voor dit onderzoek waren er in het fysieke labo twee knoppen
geplaatst op dezelfde afstand als in de virtuele omgeving. In de virtuele
omgeving waren er echter 4 knoppen met telkens een hoek van 90 graden er tussen.

In dit onderzoek heeft men ondervonden dat er 3 manieren zijn om rotationele
vervorming in te voegen:

\begin{enumerate}
    \item Als de gebruiker stil staat is het mogelijk om een kleine hoeveelheid
        constante rotatie in te voegen, de gebruiker zal dan automatisch
        meedraaien.
    \item Indien de gebruiker zelf ronddraait kan deze rotatie overdreven worden.
    \item Ten laatste is het mogelijk om het pad in de virtuele omgeving een
        bepaalde hoeveelheid te buigen die proportioneel is met de lineaire
        snelheid van de gebruiker in de fysieke omgeving.
\end{enumerate}

Deze gegevens werden dan gebruikt om de gebruiker in de richting van de volgende
knop te sturen zodat de gebruiker voor een fysieke knop stond als dit ook het
geval was in de virtuele omgeving.

In een ander onderzoek van Neth et. al.\cite{neth12} werd verder onderzocht wat
de precieze relatie tussen bewegingssnelheid en maximaal acceptabele rotationele
vervorming is.

Uit dit onderzoek bleek dat tragere snelheden grotere vervormingen toelaten, maar
dat deze relatie niet lineair is.

Aangezien mensen over het algemeen trager wandelen in virtuele omgevingen
\cite{mohler07}, kan men dit toepassen in redirected walking.

In een tweede experiment\cite{neth12} werd onderzocht of deze bevindingen
effectief konden gebruikt worden dynamische schalering van vervorming toe te
passen in een rijke virtuele omgeving. Er werd hier onderzocht of er een
significant verschil is tussen de effectiviteit van statische rotationele
vervorming versus dynamische rotationele vervorming.

Naast de technieken van Razzaque, Kohn en Whitton\cite{kohn01} om rotationele
vervorming in te voegen met een constant, een statisch en een dynamisch component,
werd er ook gebruik gemaakt van versterking van de effecten nabij de randen van
de fysieke omgeving, om het verlaten van het tracking gebied te vermijden.

Er werd gemeten wat de mediaanafstand was die een gebruiker kan wandelen in een
grote virtuele omgeving voor de gebruiker moet geforceerd geherori\"enteeerd
worden richting het centrum van de fysieke omgeving om het verlaten van het
tracking gebied te voorkomen.

Uit dit experiment is gebleken dat er een positief significant verschil is tussen
de mediaan gewandelde afstand bij dynamische vervorming versus statische
vervorming zonder een significante verhoging in simulatorziekte.

Alle vorige studies behandelden rotationele vervorming in beweging, dus over 
curves. Het is echter ook mogelijk om rotationele vervorming in stilstand te
hebben.

In\cite{steinicke09} werd bepaald dat in dit specifieke geval compressies tot
77\% acceptabel zijn voor de gebruiker het merkt.

Rotationele vervorming vormt de basis van redirected walking, gegeven een
voldoende grote ruimte kan deze techniek toegepast worden om elke virtuele
omgeving te doorlopen.


\subsubsection{Translationele vervorming}
Naast rotationele vervorming is het ook mogelijk om de lineaire snelheid van een
gebruiker te vervormen daar onderzoek heeft aangetoond dat gebruikers in virtuele
omgevingen afstand\cite{loomis03}, snelheid\cite{banton05} en afgelegde afstand
\cite{frenz07} onderschatten.

In\cite{steinicke09} werd er een experiment uitgevoerd om onder andere te bepalen
wat de acceptabele translationele vervorming is.

Er werd bepaald dat de vervorming merkbaar is bij een versnelling tussen 20\% en
60\% met een ideale waarde van ongeveer 20\%.

Translationele vervorming vormt samen met rotationele vervorming een ideale basis 
om realistische redirected walking toe te passen, maar toch de benodigde fysieke 
ruimte tot een minimum te houden.


%%%%%%%%%%%%%%%%%%%%%%%%%%%%%%%%%%%%%%%%%%%%%%%%%%%%%%%%%%%%%%%%%%%%%%%%%%%%%%%%%
\subsection{Reori\"entatietechnieken (ROTs)}
ROTs zijn technieken om te voorkomen dat de gebruiker het fysiek trackbare gebied
verlaat.

In tegenstelling tot RDTs zijn deze technieken niet noodzakelijk onmerkbaar, ik
beschrijf er hier enkele.


\subsubsection{Verbale commandos}
\todo[inline]{Verbale commandos zoals in kohn01 beschrijven}
% Om de immersie in de virtuele omgeving te verbeteren werd er ook gebruik gemaakt
% van positionele audio met met gesloten koptelefoons die het oor volledig omringen.

% Ten laatste werd er om botsingen te voorkomen een soort alarmsysteem ingevoerd
% dat gebruiker vroeg om links en rechts te draaien, zodat het systeem zich kon 
% hercalibreren.


\subsubsection{Visuele commandos}
\todo[inline]{Visuele commandos zoals in neth12 beschrijven}


\subsubsection{Afleiders}
\todo[inline]{Avatars etc...}


%%%%%%%%%%%%%%%%%%%%%%%%%%%%%%%%%%%%%%%%%%%%%%%%%%%%%%%%%%%%%%%%%%%%%%%%%%%%%%%%%
\subsection{Inherente zwaktes}
Hoewel redirected een realistische en immersieve omgeving kan cre\"eren zijn er
toch enkele inherente zwaktes waar op zich niet omheen kan gewerkt worden.

Zo is het bijvoorbeeld onmogelijk om arbitraire fysieke collisie overeen te laten
komen met collisie in de virtuele omgeving tenzij het labo expliciet voor die
virtuele omgeving is gebouwd.

Een ander voorbeeld is dat het onmogelijk is om heuvels of ander ruw terrein te
hebben in de virtuele omgeving zonder immersie te breken.

Desondanks deze behoorlijke beperkingen heeft redirected walking toch veel
potenti\"ele toepassingen zoals virtuele rondleidingen.


%%%%%%%%%%%%%%%%%%%%%%%%%%%%%%%%%%%%%%%%%%%%%%%%%%%%%%%%%%%%%%%%%%%%%%%%%%%%%%%%%
\subsection{Verder onderzoek}
\todo[inline]{Mijn BPROEF bespreken.}


\newpage
\bibliographystyle{babplain-fl}
\bibliography{refs}
\end{document}