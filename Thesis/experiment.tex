In dit onderzoek heb ik er voor gekozen om te kijken hoe effectief ``change
blindness'' is op het behouden van immersie in een virtuele omgeving in een
zeer gelimiteerde fysieke omgeving. Voor deze bachelorproef zal ik een 
experiment uitvoeren om te kijken hoe merkbaar change blindness met afleiders en 
taken is in een virtuele omgeving.


\section{Pilot study}
Nadat de virtuele omgeving volledig ontworpen was, heb ik aan enkele personen
gevraagd om deze op een gewoon computerscherm met toetsenbord en muis te
doorlopen. Dit om te kijken of de implementatie functioneel in order was, en
ook om informeel te kijken hoe goed de illusie werkt. Uit deze informele studie 
is gebleken dat niemand van mijn 7 testers de redirectie merkte. Van de 7 testers
klonk er bij 4 zelfs ongeloof over mijn verklaring dat de deuren bewogen en werd
er verzocht om de omgeving nog eens te mogen doorlopen.


\section{Ontwerp studie}
Om een objectief beeld te krijgen van de effectiviteit van mijn implementatie van
change blindness heb ik een simpel experiment opgesteld. Elke testpersoon kreeg
een briefing waar hem verteld werd dat hij door een virtueel kantoor zou gaan
wandelen. Ik deelde hem mee dat hij in elk kantoor de blinden zou moeten sluiten 
met de knop naast het raam, en dat hij de foto boven deze knop moest onthouden.
Vervolgens werd hem gevraagd het kantoor te verlaten en de deur achter hem te
sluiten en dit proces voor 3 kantoren te herhalen. Er werd ook kort aan de 
proefpersonen verteld hoe ze moesten interageren met de virtuele omgeving.

Vervolgens werd er gevraagd of er onduidelijkheden waren en begaven we ons naar 
de startpositie, en werd de proefpersoon de Oculus Rift aangeboden om zelf op te 
zetten. Na de uitvoering van het experiment werd de proefpersoon een korte 
vragenlijst voorgelegd.


\subsection{Vragenlijst}
De vragenlijst bestond uit een informatieblok en vijf vragen. Ze was beschikbaar
in het Nederlands en het Engels. De Nederlandse versie van de vragenlijst is
ingevoegd in Bijlage \ref{vragenlijst}. In het informatieblok wordt er gevraagd 
naar de leeftijd en het geslacht van de proefpersoon, dit om potentieel te 
analyseren of er verschillen in effectiviteit tussen deze groepen zijn.

Vervolgens wordt er gevraagd om een schema van het grondplan te tekenen, om te
zien of er ondanks de onmogelijke ruimte toch een consistent mentaal beeld kon
gevormd worden. Daarna wordt er gevraagd om de drie fotos op te noemen, deze
vraag is niet verwerkt in de resultaten daar deze een onderdeel uitmaakt van de
afleiders. In vraag drie moet de proefpersoon aan acht stellingen een score 
toekennen van 1 tot 6 waar 1 betekent dat hij het niet heeft gemerkt, en 6 dat 
het heel duidelijk wel is gebeurd:

\begin{enumerate}
  \item Ik zag de virtuele omgeving groter of kleiner worden.
  \item \emph{Ik voelde alsof ik het zelfde pad aan het belopen was.}
  \item Ik zag de virtuele omgeving flitsen.
  \item \emph{Ik merkte dat iets in de omgeving zich van plaats had veranderd.}
  \item Ik zag de virtuele omgeving roteren.
  \item Ik voelde mezelf groter of kleiner worden.
  \item Ik voelde me alsof ik bewogen werd.
  \item Ik merkte dat iets in de virtuele omgeving groter of kleiner werd.
\end{enumerate}

Enkel de schuingedrukte stellingen zijn echt gebeurd, de andere stellingen zouden
een zeer lage gemiddelde score moeten hebben. De laatste twee vragen bestaan uit 
een vraag waar wordt gevraagd voor algemene qualitatieve feedback over de 
immersie, en een vraag waar wordt gevraagd om de bewogen voorwerpen te 
identificeren.

De vragenlijst is gebaseerd op de vragenlijst uit het eerdere experiment van Evan
A. Suma\cite{suma11}.


\section{Uitvoering}
Het experiment is uitgevoerd met 17 proefpersonen, 4 vrouwen en 13 mannen.
Leeftijden varieerden van 21 tot 47 jaar oud met een gemiddelde van 30. Slechts
3 personen hadden geen spelervaring. Omdat zowel de groep van vrouwen als de
groep van mensen zonder spelervaring te klein is kunnen deze groepen helaas
niet apart bekeken worden. Maar op het eerste zicht lijkt er geen significant
verschil te zijn in de resultaten van deze groepen.