In dit proefschrift heb ik een onderzoek naar de effectiviteit van change
blindness beschreven, een techniek om door een grote virtuele omgeving te
wandelen in een veel kleinere fysieke omgeving. En de invloed van taken op
deze effectiviteit. De resultaten van dit onderzoek duiden er op dat mijn 
implementatie op ongeveer de helft van mijn testgroep werkte. Maar dat de het 
toch eerder niet merkbaar was dat de proefpersoon in cirkels was aan het lopen.


\section{Verder onderzoek}
Daar de effectiviteit van deze implementatie een beetje aan de lage kant is, is 
het in eerste instantie een goed idee om het experiment nog eens uit te voeren om
te kijken of het hier gaat om een probleem met mijn implementatie of een andere
factor. Verder is het mogelijk om een vergelijkende studie te doen over het 
effect van taken, met verschillende taken die bijvoorbeeld actief (indrukken van 
een knop) of passief (onthouden van een foto) uitgevoerd moeten worden.

Verder zou het mogelijk zijn om change blindness met andere technieken te
combineren zoals rotationele en dynamische translationele vervorming zoals
besproken in de literatuurstudie. Indien alle beschikbare technieken worden 
gecombineerd zou het zelfs mogelijk kunnen zijn een algemeen toepasbare 
redirectietechniek te ontwikkelen die met elke arbitraire virtuele omgeving en 
fysieke ruimte werkt.


\section{Discussie}
Het is opvallend dat mijn implementatie vergeleken met eerder onderzoek zoals dat 
van Evan A. Suma \cite{suma11} minder effectief lijkt. Ik kan enkel vermoedens
voorleggen betreffende de oorzaak hiervan. Ik vermoed dat het komt wegens
een defect in mijn opstelling zoals een gebrek aan camera's waardoor er blinde 
``spots'' zijn in de tracking area die de immersie breken. Het zou ook kunnen 
komen door een immersieprobleem zoals de laptop die door de uitvoerder van het 
experiment moet rondgedragen worden, misschien zou het toevoegen van een rugzak
met alle apparatuur er in het in dit opzicht verbeteren. Het zou ook kunnen dat 
mijn veranderingen te extreem zijn vergeleken met het eerder uitgevoerde 
onderzoek van Evan A. Suma \cite{suma11}. In dat onderzoek werd enkel de 
ori\"entatie van de deur in de hoek van de kamer veranderd, terwijl ik deze over 
de volledig lengte van een muur verschuif.