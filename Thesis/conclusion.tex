Na het overlopen van het eerder onderzoek heb ik gekozen om in dit proefschrift
een onderzoek te doen naar de effectiviteit van change blindness, een techniek 
om door een grote virtuele omgeving te wandelen in een veel kleinere fysieke 
omgeving. En de invloed van taken op deze effectiviteit. 

Er werd eerst een pilot study uitgevoerd met zeer positieve resultaten, waarna
hetzelfde experiment werd herhaald in een immersieve omgeving. De resultaten van 
dit onderzoek duiden er op dat mijn implementatie change blindness op meer dan de
helft van mijn testgroep werkte. En dat het voor de testgroep over het algemeen
eerder niet duidelijk was dat ze in cirkels waren aan het lopen. Er werd ook
onderzocht of de testgroep een consistent mentaal beeld kon vormen van de
virtuele omgeving, daar deze onmogelijk is wegens overlappende kamers. Op een
persoon na hadden alle personen in de testgroep het zelfde mentale beeld gevormd.

Uit de feedback bleek dat er enkele immersieproblemen waren met mijn opstelling,
zoals de resolutie van de Oculus Rift en haperingen in de tracker. Om de invloed 
van immersie te bekijken zou het experiment opnieuw kunnen uitgevoerd worden met 
een verbeterde opstelling. Er zou verder nog onderzocht kunnen worden hoeveel 
veranderingen er precies mogen zijn voor dit merkbaar is voor een proefperoon.

Change blindness blijft een techniek die zeer gespecialiseerd zal blijven omdat
de virtuele omgeving ontworpen moet worden met change blindness in gedachte, maar
als we andere redirectietechnieken zouden toevoegen kunnen de individuele kamers
van de virtuele omgeving groter gemaakt worden, zonder het tracking gebied te 
moeten vergroten. In de toekomst kan een combinatie van alle redirectietechnieken
gebruikt worden om zeer immersieve rondleidingen en virtuele omgevingen te maken.