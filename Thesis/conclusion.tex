In dit proefschrift heb ik een onderzoek naar de effectiviteit van change
blindness beschreven, een techniek om door een grote virtuele omgeving te
wandelen in een veel kleinere fysieke omgeving. En de invloed van taken op
deze effectiviteit.

De resultaten van dit onderzoek duiden er op dat deze techniek slechts in de
helft van de gevallen werkt in deze implementatie, dit strookt niet met eerder
onderzoek zoals dat van Evan A. Suma \cite{suma11} waar deze techniek juist een 
zeer goede illusie opbouwde. Vermoedelijk komt dit verschil in effectiviteit door 
een defect in mijn opstelling. Het zou kunnen komen door een gebrek aan camera's 
of een immersieprobleem zoals de laptop die door de uitvoerder van het experiment
moet rondgedragen worden.

Alternatief is het mogelijk dat het eerder uitgevoerd onderzoek van Evan A. Suma
\cite{suma11} andere elementen bevatte die change blindness daar effectiever
maakte. Zo werd in dat onderzoek gewoon de ori\"entatie van de deur in de hoek
van de kamer gedraaid, terwijl ik deze over de volledig lengte van een muur
verschoof.


\section{Verder onderzoek}
Om na te gaan of de ineffectiviteit een fundamenteel probleem is met mijn
uitvoering of een ander probleem is het in eerste instantie een goed probleem
om het experiment nog eens uit te voeren.

Indien hierna blijkt dat het toch goed werkt is het mogelijk om een vergelijkende
studie te doen over het effect van taken, met verschillende taken die
bijvoorbeeld actief (indrukken van een knop) of passief (onthouden van een foto)
uitgevoerd moeten worden.

Verder zou het mogelijk zijn om change blindness met andere technieken te
combineren zoals rotationele en dynamische translationele vervorming zoals
besproken in de literatuurstudie.

Indien alle beschikbare technieken worden gecombineerd zou het zelfs mogelijk
kunnen zijn een algemeen toepasbare redirectietechniek te ontwikkelen die met
elke arbitraire virtuele omgeving en fysieke ruimte werkt.