In dit onderzoek heb ik er voor gekozen om te kijken hoe effectief ``change
blindness'' is op het behouden van immersie in een virtuele omgeving in een
zeer gelimiteerde fysieke omgeving. Bij andere redirectietechnieken zoals
rotationele of translationele vervorming worden er constant veranderingen
gemaakt aan het pad dat de proefpersoon neemt. Dit lijdt tot conflicten tussen
het visuele en het vestibulaire systeem, wat op zijn beurt misselijkheid als
effect kan hebben.

Change blindness heeft dit probleem niet daar het niet het virtuele pad van de
proefpersoon is dat wordt vervormd, maar de virtuele omgeving zelf. En door de
verandering van de virtuele omgeving wordt het pad van de proefpersoon 
gemanipuleerd om binnen de beschikbare fysieke ruimte te blijven.

Voor deze bachelorproef zal ik een experiment uitvoeren om te kijken hoe 
merkbaar change blindness met afleiders en taken is in een virtuele omgeving.

Verder heb ik er voor gekozen om taken en afleiders toe te passen met als doel
de change blindness minder te laten opvallen.