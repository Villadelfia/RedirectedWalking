\section{Meeting 1: 2014-01-31}
Aanwezigen: Steven Maesen, Randy Thiemann.

In deze eerste meeting hebben we besproken hoe de bachelorproef van start zal
gaan.

Er werd afgesproken dat ik een literatuurstudie zal doen van een aantal geleverde
papers met verder nog enkele zelf opgezochte papers, dit om een onderwerp te
vinden om de bestaande research verder te zetten.

Ik zal deze literatuurstudie bij de volgende meeting voorleggen.


\section{Meeting 2: 2014-02-13}
Aanwezigen: Steven Maesen, Randy Thiemann.

Er werd in deze meeting kort overlopen wat er verbeterd moest worden aan de
al geleverde literatuurstudie. Ik zal deze herschrijven met minder lange zinnen 
en een betere verdeling van paragrafen. Ook zal ik meer aandacht besteden aan het
deel over afleiders voor reoriëntatietechnieken.

Voor het concrete onderwerp van de bachelorproef heb ik een onderzoek naar de
effectiviteit van taakgebonden afleiders in reoriëntatie en veranderingsblindheid
voorgesteld. Concreet zou dit een gang in een kantoorgebouw zijn met als taak de
blinden in elk kantoor te sluiten (waarna de deur van locatie wisselt) en bij het
buiten komen de deur te sluiten (hier gebeurt dan reoriëntatie).

Concreet zal ik nu eerst de komende weken Unity leren en een ruwe versie van dit
scenario uitwerken. Deze eerste versie zal bestuurd worden met toetsenbord en de
pitch/yaw/roll van de oculus rift.


\section{Meeting 3: 2014-02-28}
Aanwezigen: Steven Maesen, Randy Thiemann.

Er werd in deze meeting kort besproken wat het geleverde werk tot dit punt is.
Met name, ik heb de omgeving klaar met uitzondering van kamervulling en ik heb
een implementatie van de testen voor change blindness.

Ik heb ook enkele ideeën getoond van hoe ik de overblijvende problemen wil
oplossen. Ik wilde het probleem van hoogfrequent ruis oplossen met een gewogen
gemiddelde van de twee sensors, maar er werd me ook aangeraden om dit op te
lossen met een andere sensor fusion methode. Ten laatste hebben we ook een beetje
zitten na te denken over een goed systeem voor collisions in de virtuele 
omgeving, ik heb gezegd dit voorlopig te willen negeren en als laatste stap te 
houden.

Steven zal nu het trackinglokaal voor me boeken en me iets laten weten wanneer
dit beschikbaar is. Tot dat lokaal beschikbaar is zal ik verderwerken aan de
kamervulling en de literatuurstudie.


\section{Meeting 4: 2014-03-12}
Aanwezigen: Steven Maesen, Randy Thiemann.

Ik heb vandaag kort de voortgang in de aankleding van het kantoor getoond. Er 
werd hier opgemerkt dat ik best enkele verschillende prefabs zou kunnen maken
voor de kamers voor variëteit.

Achteraf zijn we de trackingruimte gaan bekijken om te bespreken hoe en wanneer
ik daar het best zou kunnen werken rond de basisintegratie van het optitrack
systeem in mijn virtuele omgeving. Ik heb gezegd vanaf maandag middag daar te
beginnen met het werk aan de integratie.

Ik heb afgesproken zelf contact te nemen om een afspraak te regelen als de
integratie compleet is. Dit houdt in dat ik een werkende kalibratie en mapping
heb tussen het systeem en mijn virtuele omgeving. Daarna zal ik beginnen met
sensor fusion tussen mijn Oculus Rift en het Optitrack systeem.

\section{Meeting 5: 2014-06-13}
Aanwezigen: Steven Maesen, Randy Thiemann.

Er werd in deze meeting de draft van mijn thesis besproken. Wegens de grote
hoeveelheid resterend werk, werd er overeen gekomen dat ik mijn verdediging op de
tweede zit zou doen.

Er werd afgesproken dat ik een tweede draft zou insturen in de loop van Juli,
ik heb later laten weten dat ik deze rond de 21ste zou insturen.